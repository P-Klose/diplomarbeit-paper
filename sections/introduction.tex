\author{Mona Angerer}

Seit dem Jahr 20xx verfügt die HTL Leonding über eine Website, die damals ebenfalls von SchülerInnen entwickelt wurde und auf Wordpress basiert. 
In den vergangenen Jahren haben sich nicht nur die technischen Möglichkeiten, sondern auch die Designstandards erheblich weiterentwickelt. 
In diesem Zeitraum traten zudem einige Probleme auf, und es wurden Verbesserungsvorschläge laut. Aufgrund dieser Veränderungen und Herausforderungen 
schlug die Schulleitung vor, im Rahmen einer Diplomarbeit einen neuen Webauftritt zu gestalten.

Die Vorgabe bestand darin, als Backend das headless CMS Storyblok zu verwenden. Die Motivation hinter dieser Entscheidung 
lag in der aktuellen Relevanz und Flexibilität dieses Content Management Systems. Die SchülerInnen wurden dazu ermutigt, 
sich eingehend über verschiedene Frontend-Varianten zu informieren und dazu zu recherchieren, um diejenige zu finden, 
die sich am besten mit Storyblok kombinieren lässt und den Anforderungen am besten gerecht wird.

Um ein breites Spektrum an Technologien abzudecken und unterschiedliche Ansätze zu fördern, wurden zwei Teams mit dieser Diplomarbeit betraut. 
Diese Herangehensweise ermöglicht es, verschiedene Ideen und Lösungsansätze zu erforschen und so eine fundierte Grundlage für die Gestaltung 
des neuen Webauftritts der HTL Leonding zu schaffen.
