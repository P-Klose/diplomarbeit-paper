
Die HTL Leonding-Website, die 20xx von Schülern auf Wordpress-Basis entwickelt wurde, 
zeigt eine durchaus den Anforderungen einer Schulhomepage entsprechenden, 
jedoch in die Jahre gekommene Plattform. Die technischen Möglichkeiten und Designstandards haben 
sich in den letzten Jahren erheblich weiterentwickelt, was dazu führt, dass die aktuelle 
Website nicht mehr den zeitgemäßen Ansprüchen entspricht. Es sind verschiedene Probleme und 
Optimierungsmöglichkeiten aufgetreten, die eine Überarbeitung erforderlich machen.

Die bestehende Website weist Unstimmigkeiten in der Benutzerführung auf, 
insbesondere im Hinblick auf die Menüstruktur, die als unübersichtlich wahrgenommen wird. 
Usern fällt es schwer, sich auf der Oberfläche zurechtzufinden und den gesuchten Inhalt auf Anhieb zu finden. 
Während mancher Content nur schwer zu finden ist, weist die Homepage auch über Inhalte auf, 
die an mehreren Stellen und unterschiedlichen Unterseiten zu finden ist. Dies verstärkt zusätzlich 
die Verwirrung und schlechte intuitive Handhabung. Zudem verfügt sie über lange Ladezeiten, was die 
Benutzererfahrung deutlich beeinträchtigt. Durch die langen Unterbrechungen, 
die in der Bedienung entstehen könnten und die immer kürzer werdende Aufmerksamkeitsspanne und Ungeduld der Menschen, 
könnte es nicht nur zu einer getrübten Stimmung, sondern sogar zum Verlassen der Website kommen. Das Design erscheint 
zu bunt und durch die vielen Bilder Videos, die oftmals einen Großteil der Seite einnehmen, 
wirkt der Webauftritt der HTL Leonding überladen und nicht mehr zeitgemäß. Des Weiteren folgt die Webanwendung einem strikten
Box-Design, was fehlende Dynamik zur Folge hat und eintönig wirkt. Auch im Mobile-Modus gibt es zudem Herausforderungen,
wie schwierige Bedienung des Menüs und Schwierigkeiten beim Zurechtfinden und Navigieren. Da immer mehr Menschen eher auf 
ihren Mobilgeräten und nicht nur ihren PCs und Laptops Webseiten aufrufen, gewinnt dies progressiv an Bedeutung.

Es hat sich gezeigt, dass die Website möglicherweise nicht mehr effektiv die Bedürfnisse und 
Ziele der verschiedenen Nutzergruppen erfüllt. Die Schulleitung hat angesichts dieser Erkenntnisse den Vorschlag gemacht, 
im Rahmen einer Diplomarbeit einen neuen Webauftritt zu gestalten. Dies bietet die Chance, die bestehenden 
Herausforderungen zu adressieren, die Website zu modernisieren und eine verbesserte Benutzererfahrung für alle Zielgruppen zu schaffen.
