Die vorliegende Diplomarbeit beschäftigt sich mit dem Neuentwurf und der Neuimplementierung der bisherigen HTL Leonding Website. 
Der Schwerpunkt liegt dabei auf der Umstrukturierung der Daten, um eine möglichst benutzerfreundliche Oberfläche zu schaffen, die für 
User aller Altersgruppen zugänglich ist. Das Design der neuen Homepage zeichnet sich durch großzügigen Whitespace, eine durchgehende 
Farbpalette sowie eingebundene Animationen und Grafiken aus. Für das Backend wurde das headless CMS Storyblok gewählt, das eine flexible 
und effiziente Verwaltung der Inhalte ermöglicht.
Insgesamt zeigt die Diplomarbeit, dass der Neuentwurf der HTL Leonding Website nicht nur eine ästhetische Verbesserung darstellt, sondern 
auch die Benutzererfahrung deutlich optimiert. Durch die klare Strukturierung der Daten und die Integration moderner Designelemente wird 
den Besuchern eine ansprechende und intuitive Plattform geboten, auf der sie leicht navigieren und relevante Informationen finden können.
Die Verwendung von Storyblok als Backend ermöglicht zudem eine effiziente Verwaltung und Aktualisierung der Inhalte, was die langfristige Pflege der Website erleichtert.
Die Diplomarbeit kann als erfolgreicher Beitrag zur Modernisierung und Verbesserung der HTL Leonding Website betrachtet werden, 
der sowohl ästhetische als auch funktionale Aspekte berücksichtigt und eine optimale Benutzererfahrung gewährleistet.