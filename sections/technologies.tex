\section{Storyblok}
\setauthor{Peter Klose}

Storyblok ist ein Headless Context Management System (CMS), welches von einem Absolventen unserer Schule mitentwickelt worden ist.
Dessen Verwendung ist auch die einzige Vorgabe, die seitens der Schule für die Umsetzung der Dipolarbeit gestellt wurde. 
Ein reguläres CMS, unter anderem Wordpress, mit dem die bisherige HTL-Website umgesetzt wurde, ist eine Softwareprodukt, 
welches den Benutzern ermöglicht, digital die Daten deren Webseite zu erstellen, gemeinsam zu bearbeiten, zu speichern 
und anschließend zu veröffenlichen.
Storyblok hingegen ist kein herkömmliches CMS, sondern ein headless, wortwörtlich aus dem englischen übersetzt "kopfloses" CMS, 
was bedeutet, dass Storyblok nur die Daten und Datenspeicherung betrifft (der Body), nicht aber die Anwendung selbst, die wir als User sehen, in unseren Fall die Webseite (der Head).
Durch diese Struktur und der daraus resultierenden Abgrenzung haben die Entwickler sehr viel Freiheit. Diejenigen, die den Content bearbeiten und erstellen müssen sich keine Gedanken über die Endgeräte machen.
Weiters sind die Frontend-Entwickler nicht limitiert auf eine bestimmte Technologie. Somit kann eine Marke oder ein Produkt per iOS App, Android App und Web representiert werden, ohne, 
dass man sich auf eine bestimmte Technologie fokusieren muss. Jeder kann nativ beziehungsweise mit der Software, die ihm am meisten zusagt, programmieren. \cite{storyblok}
\section{Next.Js 13}
\setauthor{Peter Klose}
Als Frontend-Technologie stützen wir uns auf Next.js 13, ein React Framework für full-stack Web-Applikationen. \cite{nextjsdocs}
Die hauptsächlichen Features von Next sind dabei folgende:

\subsection{Routing}
\setauthor{Peter Klose}
Next liefert direkt einen Filesystem basierten Router mit. Dieser unterstützt unter anderem Layouts, verschachtelte Routen, den Ladestatus und viele weitere.
Für die HTL-Website wurde dabei der neue App Router Ansatz gewählt, welcher mit Next 13 veröffenlicht wurde. Durch ihn werden die Neuheiten von React, wie Server Componentes und Streaming in Next eingebaut. \cite{nextjsdocsrouting}

\subsection{Optimierung}
\setauthor{Peter Klose}
In der Webbranche ist Pagespeed mitunter das Wichtigste. Dies hat auch Next verstanden und bietet daher einige Optimierungsoptionen bei Bildern, Links oder auch bei den Metadaten und Scripts. 
Bei den Bildern wurde das <img> Element dahingehend überarbeitet, dass Bilder "lazy" geladen werden und dynamisch auf die große des Bildschirms angepasst werden. 
Das <a> Tag beziehungsweise der Link wurde dahingehend angepasst, dass die Webseite, wenn sie vollständig geladen ist, die möglichen Ziele vorlädt um somit eine schnelleren und weicheren Wechsel der Seiten zu ermöglichen. \cite{nextjsdocsoptimizations}

\subsection{Rendern}
\setauthor{Peter Klose}
Next bietet auch die Möglichkeit der Client oder Server-Components. Client-Components sind die Teile der Webseite, die im Client, also im Browser, geladen und gezeigt werden.
Server-Components hingegen müssen zunächst auf dem Server zusammengebaut werden und werden dann erst zum Client geschickt. \cite{nextjsdocsrendering}

\section{Vercel}
\setauthor{Peter Klose}
Beim Deployment Vercel hauptsächlich verwendet, weil Next aus dem selben Hause stammt und das Deployment somit keine Problem darstellt.
Sobald 'next build' funktioniert, kann das GitHub Repository einfach mit einem Vercel Account verbunden werden und der main-Branch ist nach 2 Minuten deployed. \cite{vercel}

\section{TailwindCss}
\setauthor{Peter Klose}
Als CSS Framework wird TailwindCSS als PostCSS Plugin verwendet. 
Um trotzdem eine übersichtliche Struktur in die Klassennamen beizubehalten wurde die VS-Code Erweiterung Prettier verwendet. 
Dadurch werden die Klassennamen gemäß der vorgeschlagenen Sortierung von Tailwind geordnet.


\section{Framer Motion}
\setauthor{Peter Klose}
Für die Umsetzung der Frontend-seitigen Animationen auf der Website wurde das fortschrittliche Animationsframework 
Framer Motion ausgewählt und implementiert. Framer Motion bietet eine leistungsstarke und flexible Lösung für 
die Erstellung von dynamischen und ansprechenden Animationen, die den HTML- und CSS-Code der Website betreffen.

Im Gegensatz zu herkömmlichen CSS-Animationen bietet Framer Motion ein breites Spektrum an fortschrittlichen 
Animationstechniken und -funktionen, die weit über einfache Transitions und Effekte hinausgehen. Eines der herausragenden 
Merkmale dieses Frameworks sind die Layout-Animationen, die es ermöglichen, komplexe seitenübergreifende Transformationen zu realisieren. 
Diese Layout-Animationen bieten die Möglichkeit, den Inhalt und die Struktur der Website dynamisch und ansprechend zu gestalten, 
wodurch ein flüssiges und nahtloses Benutzererlebnis geschaffen wird.

Darüber hinaus bietet Framer Motion spezialisierte Funktionen für Load- und Unload-Animationen. Diese Animationen sind darauf 
ausgerichtet, das Laden und Entladen des DOM (Document Object Model) zu animieren, um fließende und eindrucksvolle Seitenübergänge 
zu ermöglichen. Dies trägt nicht nur zur Verbesserung der visuellen Darstellung und Benutzerfreundlichkeit der Website bei, sondern 
erhöht auch das Engagement der Benutzer und optimiert die Gesamtusability.

Die vielseitigen und erweiterten Funktionen von Framer Motion ermöglichen es, die Animationen und Interaktionen auf der 
Website gezielt und präzise zu steuern. Sie bieten eine hohe Flexibilität und Anpassungsfähigkeit, wodurch individuelle und 
maßgeschneiderte Animationseffekte leicht umgesetzt werden können. Insgesamt stellt die Verwendung von Framer Motion als 
Animationsframework eine effiziente und innovative Lösung dar, um die Website mit ansprechenden, dynamischen und benutzerfreundlichen 
Animationen zu bereichern und das Benutzererlebnis auf ein neues Niveau zu heben.

\section{Lottie}
\setauthor{Angerer Mona}
Als zentrales SVG-Animationsframework für die Implementierung der animierten Grafiken auf der Website wird Lottie eingesetzt. 
Dieses Framework bietet eine leistungsstarke Lösung für die Darstellung von komplexen und ansprechenden Animationen direkt im Webbrowser, 
ohne die Ladezeiten oder die Performance der Website negativ zu beeinflussen.

Die Integration von Lottie beginnt bereits im Exportprozess aus Adobe After Effects. Dort werden die Animationen speziell als Lottie JSON 
exportiert, wodurch eine nahtlose und fehlerfreie Kompatibilität zwischen den beiden Programmen gewährleistet wird. Dieser Exportprozess 
ermöglicht es, die in After Effects erstellten Animationen direkt in Lottie-kompatible JSON-Dateien umzuwandeln, die dann problemlos in die 
Website eingebunden werden können.

Nach dem Export erfolgt die weitere Implementierung der animierten Grafiken im Code durch die Einbindung dieser JSON-Dateien 
in den Lottie Player. Der Lottie Player ist ein spezialisiertes JavaScript-Plugin, das die Darstellung der Lottie JSON-Dateien im 
Webbrowser steuert. Durch diese Kombination werden alle übrigen Animationen, die nicht durch Framer Motion oder andere Technologien 
gesteuert werden können, effizient und zuverlässig durch Lottie gesteuert.

Diese Vorgehensweise bietet den Vorteil, dass komplexe und detailreiche Animationen, die mit Adobe After Effects erstellt wurden, 
direkt und ohne Qualitätsverlust auf der Website dargestellt werden können. Zudem ermöglicht die Verwendung von Lottie eine höhere 
Flexibilität und Anpassungsfähigkeit der Animationen, da sie leicht aktualisiert oder modifiziert werden können, ohne den gesamten Code 
oder die Website neu zu laden.

