Wie im Kapitel Design bereits beschreiben, haben wir um die Website aufzulockern einige Animationen eingebaut. 
In folgendem Abschnitt wird erläutert, mit welchem Verfahren diese ausgewählt wurden, welche Technolgien verwendet wurden und die 
wie die Einbindung auf der Homepage funktioniert.

\section{Auswahl der Animationen} \label{sec:Recherche}
\setauthor{Angerer Mona}

Bei der Auswahl der eingebundenen Animationen wurde großer Wert auf die Benutzerfreundlichkeit und Navigation gelegt.
Deswegen wird beim Aufruf der neuen Homepage direkt mit einer etwa zwei-sekündigen Startanimation begonnen. Diese soll die 
Aufmerksamkeit der User erlangen und einen spannenden Start bieten. Sie startet mit dem HTL Leonding Logo und dem Sschriftzug "HTL Leonding next level" und nachdem
das Logo von der linken Seite zur rechten Seite schwebt und der Schriftzug veschwindet verwandelt sich das Logo in eine Grafik des 
HTL Gebäudes aus der Vogelperspektive. Anschließend scrollen die wichtigsten News-Beiträge von rechts aus dem Gebäude. 
Auch wurde für jede Abteilung der HTL eine eigene kleine Animation gefertigt, die die jeweilige Fachriochtung repräsentiert.


\section{Erstellung} \label{sec:Recherche}
\setauthor{Angerer Mona}

Nachdem die Grafiken sorgfältig und präzise mittels Adobe Illustrator erstellt wurden, bietet Adobe After Effects die Möglichkeit, 
diese Grafiken zu importieren und mit spezialisierten Animationstechniken zu veredeln. In After Effects können verschiedene Animationseffekte, 
Übergänge und Bewegungen hinzugefügt werden, um den Grafiken Leben einzuhauchen und sie dynamischer und ansprechender zu gestalten. 
Dieser Animationsprozess ermöglicht es, komplexe Bewegungen und Interaktionen zu erstellen, die die visuelle Darstellung der Grafiken 
erheblich verbessern.

Nach Abschluss der Animation in After Effects erfolgt der Export der animierten Grafiken im JSON-Format, welches von LottieFiles 
unterstützt wird. LottieFiles ist ein leistungsstarkes Tool, das speziell für die Integration von animierten Grafiken in Web- und 
App-Designs entwickelt wurde. Es ermöglicht die nahtlose Einbindung der animierten Grafiken in verschiedene Projekte und Plattformen, 
wobei die hohe Qualität und flüssige Darstellung der Animationen beibehalten wird. Durch die Verwendung von LottieFiles können die 
animierten Grafiken effizient in die Website integriert werden, ohne die Ladezeiten oder die Leistung der Website zu beeinträchtigen.


\section{Einbindung} \label{sec:Recherche}
\setauthor{Angerer Mona}


\section{Schwierigkeiten} \label{sec:Recherche}
\setauthor{Angerer Mona}


